\documentclass[12pt]{jsarticle}
\usepackage{listings}
\begin{document}
確率の前提知識

ベイズ則は,次の式で表される.

\begin{equation}
  P(B|A) = \frac{P(A|B)P(B)}{P(A)}
\end{equation}

\begin{equation}
 P(z_i,_t){t_i}
\end{equation}

また確率の基本法則として

加法定理
\begin{equation}
  P(X)=\sum_{Y}^{}{p(X,Y)}
\end{equation}

乗法定理
\begin{equation}
  P(X,Y)=p(Y|X)p(X)
\end{equation}


Kalmanfilterはガウス分布と線形性が成り立つと仮定されている

信念belは平均$\mu_t$と共分散$\Sigma_t$で表され,マルコフ性と次の3つの条件で,事後信念はガウス分布となる.

1. 状態遷移確率$p(x_t|u_t,x_{t-1})$が,ガウス雑音を足した線形関数

2. 計測確率$p(z_t|x_t)$も線形であり,その雑音はガウス雑音

3. 初期信念$bel(x_0)$が正規分布

\begin{lstlisting}[basicstyle=\ttfamily\footnotesize, frame=single]
\begin{equation}

algorithm Kalman_filter(\mu_{t-1}, \Sigma_{t-1}, u_t, z_t):

  \bar{\mu_t} = A_t\mu_{t-1}+B_tu_t

  \bar{\Sigma_t} = A_t\Sigma_{t-1}A_t^T+R_t

  K_t = \bar{\Sigma_t}C_t^T(C_t\bar{\Sigma_t}C_t^T+Q_t)^{-1}

  \mu_t = \bar{\mu_t}+K_t(z_t-C_t\bar{\mu_t})

  \Sigma_t = (I-K_tC_t)\bar{\Sigma_t}

  return   \mu_t, \Sigma_t
\end{equation}
\end{lstlisting}

\begin{thebibliography}{9}
  \bibitem{ProbablityRobotics}Sebastian Thrun, Wolfram Burgard, Dieter Fox, "確率ロボティクス", 2015,マイナビ出版 
  \bibitem{基礎からわかる時系列分析},萩原 淳一郎,瓜生 真也,牧山 幸史,石田 基広 , "基礎からわかる時系列分析", 2018, Data Science Library
\end{thebibliography}
\end{document}
