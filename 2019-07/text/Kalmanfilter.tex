\documentclass[12pt]{jsarticle}
\begin{document}
Kalmanfilterはガウス分布と線形性が成り立つと仮定されている

またモデルについて
\begin{equation}
  X_t=K_xX_{t-1}
\end{equation}

ベイズ則は,次の式で表される.

\begin{equation}
  P(B|A) = \frac{P(A|B)P(B)}{P(A)}
\end{equation}

\begin{equation}
 P(z_i,_t){t_i}
\end{equation}

また確率の基本法則として

加法定理
\begin{equation}
  P(X)=\sigma{1}{p(X,Y)}
\end{equation}

乗法定理
\begin{equation}
  P(X,Y)=p(Y|X)p(X)
\end{equation}

\begin{thebibliography}{9}
  \bibitem{ProbablityRobotics}Sebastian Thrun, Wolfram Burgard, Dieter Fox, "確率ロボティクス", 2015,マイナビ出版 
\end{thebibliography}
\end{document}
