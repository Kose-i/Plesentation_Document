\documentclass[12pt]{jsarticle}

\usepackage{amsmath}

%\editor{tamura-kosei}

\begin{document}

\section{特異値分解(Singular Value Decomposition;SVD)}
\begin{equation}
  \label{A=USV}
  A = U \Sigma V
\end{equation}
ここで$A$を$m×m$行列で,$U$は$m×m$直交行列,$\Sigma$は$m×n$行列,$V$は$n×n$直交行列である.
$D$は対角行列であり,$d_i^*$を行列$A$の特異値とすると,
\begin{equation}
  \label{D_diag}
  D = diag(d_1^*, d_2^*, ..., d_n^*)
\end{equation}
とかける.
特異値とは,

特異値分解の性質
\begin{itemize}
  \item $A$が与えられたとき,特異値を定める行列$Σ$は一意に決まりますが,直交行列$U$,$V$は一意に定まるとは限りません。
  \item 行列の(0 でない)特異値の数は,その行列のランクと一致します。
  \item 行列の特異値の二乗和はその行列の全成分の二乗和と等しいです。先ほどの具体例ではどちらも24になっています。この値の平方根を行列のフロベニウスノルムと言います。
  \item $A$が対称行列のとき,$A$の固有値と特異値は一致します。対称行列は直交行列で対角化できるからです。
  \item $A^TA$ の 0 でない固有値の正の平方根は$A$の特異値です。これは,$A=UΣV$のとき$A^TA=V^TΣ^TΣV$であることから分かります。同様に,$AA^T$ の 0 でない固有値の正の平方根も$A$の特異値です。
\end{itemize}

\section{一般逆行列}

定義
 $A$を$(n,m)$型行列とする線形方程式$Ax=y$が解$x$をもつような$y$に対して,$x=A^-y$がこの方程式の一つの解となる場合,$(m,n)$型行列$A^-$を$A$の一般逆行列という.
ムーアペンローズ一般逆行列(Moore and Penrose generalized inverse)の定義
\begin{enumerate}
  \item $AA^-A = A$
  \item $A^-AA^- = A^-$
  \item $(A A^-)' = AA^-$
  \item $(A^- A)' = A^-A$
\end{enumerate}
 

\begin{thebibliography}{9}
  \bibitem{koukousuugaku} 特異値分解の定義,性質,具体例, https://mathtrain.jp/svd
  \bibitem{} 一般逆行列 https://www.slideshare.net/wosugi/ss-79624897
  \bibitem{} 射影行列・一般逆行列・特異値分解 
\end{thebibliography}

\end{document}
