\documentclass[12pt]{jsarticle}

\title{自己符号化器}

\begin{document}

自己符号化器とは,目標出力を伴わない,入力だけの訓練データを使った教師なし学習により,データをよく表す特徴を獲得することを目標とするニューラルネットです.ディープネットの事前学習,すなわちその重みのよい初期値を得る目的に利用される.

入力$x$がそのまま入力層のユニットの出力となり,出力層からの出力$y$が
\begin{equation}
  y = f(Wx+b)
\end{equation}
と決定されるうようなものである.
2層ネットワークは最初のそうでは上の式に従って入力$x$を$y$に変換し,次の層ではこうして得た$y$を入力$x$と同じ空間に戻す変換を行います.変換の結果を$\hat{x}$と書き,$y$から$\hat{x}$への変換を
\begin{equation}
  \hat{x} = \tilde{f}(\tilde{W}y+\tilde{b})
\end{equation}
と書くことにします.ここで$\tilde{f}$は追加した層の活性化関数ですが,一般に最初の層の活性化関数$f$とは異なっていて構わない.これら二つをまとめると\begin{equation}
  \hat{x}(x) = \tilde{f}(\tilde{W}f(Wx+b) + \tilde{b})
\end{equation}
と書ける.
最初の変換を符号化(encode),二番目の変換を復号化(decode)と呼ぶ.

入力層と中間層のユニット数をそれぞれ$D_x$と$D_y$とすると,$W$,$\tilde{W}$のサイズはそれぞれ$D_y×D_x$,$D_x×D_y$になる.重み共有では,次の関係を満たす.
\begin{equation}
  \tilde{W} = W^T
\end{equation}


\begin{thebibliography}{9}
  \bibitem{DeepLearning} 岡谷貴之, 機械学習プロフェッショナル 深層学習,2015.
\end{thebibliography}

\end{document}
